\documentclass{article}
\usepackage{colortbl}
\usepackage{hyperref}
\usepackage{xcolor}
\hypersetup{
	colorlinks = true,
	linkbordercolor = white,
	urlcolor = black,
	citecolor = black,
	allcolors = black
}

\usepackage{LectureTemplate}
\usepackage{tabularx}
\parindent=0pt

\makeatletter
\renewcommand{\@makefntext}[1]{\parindent 1em
	\noindent\hbox to 1em{}% if you want to indent footnote text you can change the width of the hbox (e.g. \hbox to 2em{})
	\llap{\if@RTL\else\latinfont\fi\@thefnmark)\,\,}#1}
\makeatother
\renewcommand\UrlFont{\color{blue}\rmfamily\itshape}

%\settextfont{XB Niloofar}
\defpersianfont\nastaliq{IranNastaliq.ttf}
\settextfont[Scale = 1.0 ,
             BoldFont = *Bd ,
             ItalicFont = *It ,
             BoldItalicFont = *BdIt ,
             Extension = .ttf
            ]{XB Yas} 


%\pagestyle{fancy}
\vspace{1cm}
\title{جلسه 3 }
\author{نگارنده: محمد حسین فرخی لاشیدانی }
%\date{}
%\vspace{-0.5cm}

\begin{document}
\handout{{\textbf{مدرس:}} دکتر مجتبی رفیعی}
{
	اصول سیستم‌های عامل
}
{
نیمسال دوم ۱۴۰۱−۱۴۰۰
}
\maketitle

\tableofcontents 
\vspace{0.75 cm}



\section{سیستم‌عامل -تعریف ابتدایی}

سیستم‌عامل نرم‌افزاری است که یک سیستم کامپیوتری را مدیریت می‌کند، به عنوان یک واسطه بین کاربر یک سیستم کامپیوتری و سخت‌افزار قرار می‌گیرد و محیطی را برای اجرای برنامه‌های کاربردی فراهم می‌کند.




\section{اهداف سیستم‌عامل}
\begin{enumerate}
    \item[•] اجرای برنامه‌های کاربر و حل کردن ساده‌تر مسائل کاربران،
    \item[•] استفادۀ راحت‌تر و ساده‌تر از یک سیستم کامپیوتری،
    \item[•] استفادۀ کارا از منابع سخت‌افزاری یک سیستم کامپیوتری.
\end{enumerate}



\section{مؤلفه‌های یک سیستم کامپیوتری }

به طور کلی یک سیستم کامپیوتری از چهار مؤلفۀ زیر تشکیل شده‌است:

\begin{enumerate}

	\item \textbf{سخت‌افزار \lr{(Hardware)}:}  منابع محاسباتی پایۀ آن عبارتند از:
	\begin{enumerate}
	    \item[•]واحد پردازش مرکزی \lr{(CPU)}،
	    \item[•]حافظه \lr{(Memory)}،
	    \item[•]دستگاه های ورودی و خروجی \lr{(I/O\; Devices)}.
	\end{enumerate}



	\item \textbf { سیستم‌عامل \lr{(Operating \; System}:} به منظور کنترل و هماهنگی در استفاده از سخت‌افزار‌های یک سیستم کامپیوتری در میان برنامه‌های کابردی و کاربران مورد استفاده قرار می‌گیرد.
    
    \item \textbf{برنامه‌های کاربردی \lr{(Application \; Programs)}:} برنامه‌هایی هستند که از منابع یک سیستم کامپیوتری برای حل یا رفع نیاز‌های یک کاربر استفاده می‌کنند، مثل کامپایلر‌ها، مرورگر‌های وب، واژه‌پرداز‌ها و ... .



	\item \textbf{کاربران \lr{(Users)}:}
	استفاده کنندگان از یک سیستم کامپیوتری هستند و متناسب با نیاز ممکن است چندین برنامۀ کاربردی را روی یک سیستم کامپیوتری اجرا کنند.
	
\end{enumerate}


شکل زیر نمای کلی مؤلفه‌های یک سیستم کامپیوتری و جایگاه مؤلفۀ سیستم‌عامل در میان دیگر مؤلفه‌ها را نشان می‌دهد:


\begin{center}
\includegraphics[scale=0.8]{Chart.jpg}   
\end{center}


\section{یادآوری: ساختمان کلی سخت‌افزار کامپیوتر}

شکل زیر اجزای کلی تشکیل دهندۀ سخت‌افزار یک سیستم کامپیوتری را نشان می‌دهد:
\begin{center}


\includegraphics[scale=0.5]{Hardware.png}   

\end{center}
%\vspace{1cm}
ارتباط بین اجزای تشکیل دهندۀ سخت‌افزار یک سیستم کامپیوتری بر اساس سه نوع جریان اطلاعات شکل می‌گیرد:

\begin{enumerate}
    \item[\textbf{1.}]\textbf{جریان دستورالعمل‌ها:}
    
    شکل زیر جریان دستورالعمل بین مؤلفه‌های ساختمان سخت‌افزار کامپیوتر را نشان می‌دهد.
لازم به ذکر است جریان دستورالعمل‌ها، میان حافظۀ اصلی و واحد محاسبه و منطق یک‌طرفه است.

\begin{center}
\includegraphics[scale=0.5]{dastoor.png}   
\end{center}

    
\item[\textbf{2.}] \textbf{جریان داده‌ها:}

    شکل زیر جریان داده‌ها بین مؤلفه‌های ساختمان سخت‌افزار کامپیوتر را نشان می‌دهد.
لازم به ذکر است جریان داده‌ها، میان حافظۀ اصلی و واحد محاسبه و منطق دوطرفه است.

\begin{center}

\includegraphics[scale=0.45]{dade.png}   
\end{center}

%    \vspace{0.5cm}
\item[\textbf{3.}] \textbf{جریان سیگنال‌های کنترلی:}

    شکل زیر جریان سیگنال‌های کنترلی بین مؤلفه‌های ساختمان سخت‌افزار کامپیوتر را نشان می‌دهد.

\begin{center}

\includegraphics[scale=0.5]{signal.png}   
\end{center}

\end{enumerate}
%\vspace{0.5cm}
\section{نقش سیستم‌عامل}
به طور کلی بر اساس دو دید عمده می‌توان نقش‌های سیستم‌عامل را در یک سیستم کامپیوتری تعیین کرد:
\begin{enumerate}
    \item[•]دید کاربر،
    \item[•]دید سیستم کامپیوتری.
\end{enumerate}
هر سیستم‌عامل متناسب با هر نقش، مجموعه‌ای از عملکردها
\lr{(Functionalities)}
 را تأمین می‌کند.

\subsection{نقش سیستم‌عامل از دید کاربر}
فراهم کردن امکاناتی برای استفادۀ ساده و راحت با کارایی خوب به همراه امنیت نسبی است. در چنین دیدی نگرانی در رابطه با بهره‌برداری از منابع یک سیستم کامپیوتری به چشم نمی‌خورد.

%
\vspace{.2cm}
\textbf{ نکتۀ جانبی:}برخی از سیستم عامل‌ها ممکن است برای کاربردهای خاصی مثل سیستم‌های پدافندی-دفاعی طراحی شده باشند که کاربر مداخلۀ مستقیم با آن‌ها را ندارد.

\subsection{نقش سیستم‌عامل از دید سیستم کامپیوتری }
برنامه‌ای است که بیشترین تعامل را با سخت‌افزار دارد و به عنوان یک تخصیص دهندۀ منابع یا اصطلاحاً 
\lr{(Resource Allocator)}
 شناخته می‌شود.




\end{document}


